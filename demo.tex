%!TEX program = xelatex
\documentclass{beamer} % Use to generate the slides for the presentation.
%\documentclass[handout,notes, gray]{beamer} % Use to print the slides and notes without animations or pauses. Gray is used to convert all color to grayscale.

\usetheme{bonbon}

\newcommand*{\bonbon}{BonBon\xspace}

\title{\bonbon}
\subtitle{A minimalistic theme for Beamer}
\author{Duncan Paul Attard}
\date{\today}


%%%%%%%%%%%%%%%%%%%%%%%%%%%%%%%%%%%%%%%%%%%%%%%%%%%%%%%%%%%%%%%%%%%%%%%%%%%%%%%%
% Package imports and basic configuration.                                     %
%%%%%%%%%%%%%%%%%%%%%%%%%%%%%%%%%%%%%%%%%%%%%%%%%%%%%%%%%%%%%%%%%%%%%%%%%%%%%%%%

% Used to display code listings.
\usepackage{listings}

% Used for drawings.
\usepackage{tikz}

% Used to insert, or not, spaces.
\usepackage{xspace}

% Used to create beautiful tables.
\usepackage{booktabs}

% Configure the style for Erlang code snippets.
\lstset{
	language=erlang,
	basicstyle=\footnotesize\ttfamily,
	aboveskip=0.2em,
	belowskip=0.2em,
	lineskip=0.1em,
	numbers=left,
	numberstyle=\tiny,
  numbersep=1em,
	xleftmargin=-0.4em,
	xrightmargin=0em,
	morecomment=[l]{\%\%},
	breaklines=false,
	showstringspaces=true,
	basewidth=0.55em,
	escapeinside={(*@}{@*)},
	moredelim=**[is][\only<+>{\color{red}}]{@}{@},
}

% Configure libraries to be imported by TikZ.
\usetikzlibrary {
  positioning
}

% Uncomment this to render everything in black and white.
% \usecolortheme{bonbonbw}

\begin{document}

\maketitle

% Show table of contents.
\begin{frame}{Table of contents}
  \setbeamertemplate{section in toc}[sections numbered]
  \tableofcontents[hideallsubsections]
\end{frame}

\section{Introduction}

\begin{frame}[fragile]{\bonbon}

  % \only<1-3| handout:0>{foo}

  \bonbon is a Beamer theme that renders content in a clean and
  minimalistic fashion.
  It can be enabled like so:
  \begin{verbatim}  \documentclass{beamer}
  \usetheme{bonbon}\end{verbatim}

  \bonbon's color scheme balances between the simplicity of the inner slide
  content and the colorful visual elements of section and slide titles.

  \bigskip
  The included black and white theme can be used to render content that is
  destined for printing.
  \begin{verbatim}  \usecolortheme{bonbonbw}\end{verbatim}

  % Slide notes.
  \note[item]{This is the first note.}
  \note[item]{A second note.}
\end{frame}

\begin{frame}[fragile]{Sections and Subsections}
  Sections and subsections can be used to group slides together:
  \begin{verbatim}  \section{My Section}  \end{verbatim}
  and
  \begin{verbatim}  \subsection{My Subsection}  \end{verbatim}

  Section slides and slide titles are equipped with a progress indicator
  in place of slide numbers.
\end{frame}

\section{Elements}

\begin{frame}{Fonts}
  All content in \bonbon is styled using the \emph{Open Sans} font, providing
  sensible defaults:
  \begin{itemize}
    \item \emph{emphasized}
    \item \textbf{bold}
    \item \textbf{\emph{bold emphasized}}
    \item \alert{alerted}
  \end{itemize}

  \texttt{Monospaced} text is styled using \emph{Cousine}:
  \begin{itemize}
    \item \texttt{\textit{italic}}
    \item \texttt{\textbf{bold}}
    \item \texttt{\textbf{\textit{bold italic}}}
  \end{itemize}
\end{frame}

\begin{frame}{Lists}
  \begin{columns}[T, onlytextwidth]
    \column{0.33\textwidth}
    Enumerations
    \begin{enumerate}
      \item First,
      \item Second and
      \item Third.
    \end{enumerate}

    \column{0.33\textwidth}
      Items
      \begin{itemize}
        \item Java
        \item Erlang
        \item Python
      \end{itemize}

    \column{0.33\textwidth}
      Descriptions
      \begin{description}
        \item[Private] Hidden.
        \item[Static] Global.
      \end{description}
  \end{columns}
\end{frame}

\begin{frame}{Blocks}
  Block environments are styled with a contrasting background color, isolating
  them from the surrounding text.

  \begin{block}{Default}
    Default blocks highlight important definitions or concepts.
  \end{block}

  \begin{alertblock}{Alert}
    Alert blocks provide advice that should't be ignored.
  \end{alertblock}

  \begin{exampleblock}{Example}
    Example blocks aid in the understanding of an idea.
  \end{exampleblock}
\end{frame}

\begin{frame}{Tables and Figures}
\begin{columns}
  \column{0.5\textwidth}
  \begin{table}
    \begin{tabular}{cl}
      \toprule
      Rank & Language\\
      \midrule
      1 & JavaScript\\
      2 & Java\\
      3 & Python\\
      4 & PHP\\
      \bottomrule
    \end{tabular}
    \caption{Top active languages, '17 (GitHub)}
  \end{table}
  \column{0.5\textwidth}
  \begin{figure}
    \begin{tikzpicture}
      \tikzset{
        component/.style={
      		font=\footnotesize,
      		align=center,
      		draw,
      		minimum width=2cm,
      		minimum height=0.6cm,
      		text depth=-0.08em,
      	},
      }
      \node[draw, component] (server) {Server};
      \node[draw, component, below=2cm of server] (client) {Client};
      \draw[-stealth] ($(client.north west)+(0.2cm, 0)$) -- ($(server.south west)+(0.2cm, 0)$) node [pos=0.5, sloped, above, font=\itshape\scriptsize]{request};
      \draw[-stealth] ($(server.south east)+(-0.2cm, 0)$) -- ($(client.north east)+(-0.2cm, 0)$) node [pos=0.5, sloped, above, font=\itshape\scriptsize]{response};
    \end{tikzpicture}
    \caption{Client-server communication}
  \end{figure}
\end{columns}
\end{frame}

\begin{frame}{Math}
  The binomial coefficient gives the number of ways $k$ can be chosen from $n$ without repetition:
  \begin{equation*}
    \frac{n!}{k!(n-k)!} = \binom{n}{k}
  \end{equation*}
\end{frame}

\section{Subsections}

\begin{frame}[fragile]{Adding breaks}
  Subsections can be used to introduce breaks in a presentation, directing
  attention to particular topics.
  \begin{verbatim}  \subsection{Quiz}\end{verbatim}

  Subsections are especially useful when used with the \texttt{subsection} frame
  flag:
  \begin{verbatim}  \begin{frame}[subsection]{Question}\end{verbatim}
  This changes styling of the frame content to match that of the subsection
  page, as seen next.
\end{frame}

% Subsections can be very useful to create breaks - I plan to use them for quiz
% during the lesson delivery.
\subsection{Quiz}

{
\begin{frame}[subsection, fragile]{Before we begin\ldots}
  Frames having the \texttt{subsection} flag set must be grouped
  using either \texttt{\{} and \texttt{\}} or \texttt{\textbackslash{begingroup}}
  and \texttt{\textbackslash{endgroup}} pairs.

  \bigskip
  Notice that the color of the frame title is set to match the background of
  the previous slide.

  \bigskip
  The subsection flag can be safely used with other flags, for instance:
  \begin{verbatim}  \begin{frame}[subsection, fragile]{Question}\end{verbatim}
\end{frame}

\begin{frame}[subsection, fragile]{Question}
  \begin{block}{What does this Erlang function do?}
  \begin{lstlisting}
  member_of(_, []) -> false;
  member_of(X, [X | _Xs]) -> true;
  member_of(X, [_ | Xs]) -> member_of(X, Xs).
  \end{lstlisting}
  \end{block}

  \bigskip
  \begin{enumerate}
    \item It does \emph{not} compile because \texttt{\_} is an illegal character
    \alert<2->{\item It checks whether an element is a member of the list}
    \item It loops indefinitely due to the lack of a base case
    \item It does \emph{not} compile because the last function clause ends with a \texttt{;}
  \end{enumerate}
\end{frame}
}

\begin{frame}{Conclusion}
  The source code for this theme and demo presentation can be obtained from:
  \begin{center}\url{github.com/duncanatt/bonbon}\end{center}
  Enjoy!
\end{frame}

\end{document}
