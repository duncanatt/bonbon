%%%%%%%%%%%%%%%%%%%%%%%%%%%%%%%%%%%%%%%%%%%%%%%%%%%%%%%%%%%%%%%%%%%%%%%%%%%%%%%%
% Package imports and basic configuration.                                     %
%%%%%%%%%%%%%%%%%%%%%%%%%%%%%%%%%%%%%%%%%%%%%%%%%%%%%%%%%%%%%%%%%%%%%%%%%%%%%%%%

% Used to determine which typesetting engine is being used. This makes it
% possible to tailor the configuration for specific engines later.
\usepackage{iftex}

% Load font packages according to the typesetting engine being used.
\ifPDFTeX
  \usepackage[T1]{fontenc}
  \usepackage{lmodern}
\else
  \ifXeTeX
    \usepackage{fontspec}
  \else
    \usepackage{luatextra}
  \fi
\fi

% Used to configure the page size and orientation.
\usepackage[a4paper,
  footnotesep=2\baselineskip,
  left=1.25in,
  right=1.25in,
  top=1.5in,
  bottom=1.5in,
  footskip=0.25in]{geometry}

% Used for formatting chapter and section titles including their spacing.
\usepackage{titlesec}

% Used to support subfigures, *without* loading the caption package,
% as this clashes with the River Publishers class file.
\usepackage[caption=false]{subfig}

% Used to wrap text around figures.
\usepackage{wrapfig}

% Used for color support.
\usepackage[table,dvipsnames]{xcolor}

% Used to import graphics.
\usepackage{graphicx}

% Used for floating figures.
\usepackage{float}

% Used to provide extra mathematical fonts and symbols.
\usepackage{amsfonts}

% Used for facilitating the writing of mathematical formulas.
\usepackage{amsmath}

% Used for mathematical theorems and definition support.
\usepackage{amsthm}

% Used for extended mathematical symbols like black square, etc.
\usepackage{amssymb}

% Used for drawing extended mathematical symbols like HML denotation brackets..
\usepackage{stmaryrd}

% Used for dingbats.
\usepackage{pifont}

% Used for drawings.
\usepackage{tikz}

% Used to draw frames around text.
\usepackage{mdframed}

% Used to insert, or not, spaces.
\usepackage{xspace}

% Used to display code listings.
\usepackage{listings}

% Used to display algorithms.
% \usepackage[linesnumbered, lined, noresetcount]{algorithm2e}

% Used for modifying enumeration items bullets and to create inline lists.
\usepackage[inline]{enumitem}

% Used to underline text that supports line breaks.
\usepackage[normalem]{ulem}

% Used to enable hyperlinks in the documents where cross-referencing occurs.
\usepackage{hyperref}

% Used to for making references include the caption label (e.g. figure, eq, etc.)
\usepackage[nameinlink]{cleveref}

% Used for numbering lines in math environments.
\usepackage[mathlines]{lineno}

% \usepackage{mathspec}
% \setallmainfonts(Digits,Latin){Georgia}


%%%%%%%%%%%%%%%%%%%%%%%%%%%%%%%%%%%%%%%%%%%%%%%%%%%%%%%%%%%%%%%%%%%%%%%%%%%%%%%%
% Extended package configuration.                                              %
%%%%%%%%%%%%%%%%%%%%%%%%%%%%%%%%%%%%%%%%%%%%%%%%%%%%%%%%%%%%%%%%%%%%%%%%%%%%%%%%

% Configure libraries to be imported by TikZ.
\usetikzlibrary{
  calc,
  arrows
}

% Configure the general style for code snippets.
\lstset{
	% language=erlang,
	basicstyle=\footnotesize\ttfamily,
	aboveskip=1em,
	belowskip=1em,
	lineskip=0.1em,
	numbers=left,
	numberstyle=\tiny,
  numbersep=1em,
	xleftmargin=1em,
	xrightmargin=0em,
	morecomment=[l]{\%\%},
	breaklines=false,
	showstringspaces=true,
	basewidth=0.55em,
	escapeinside={(*@}{@*)},
	moredelim=**[is][\only<+>{\color{red}}]{@}{@},
}


%%%%%%%%%%%%%%%%%%%%%%%%%%%%%%%%%%%%%%%%%%%%%%%%%%%%%%%%%%%%%%%%%%%%%%%%%%%%%%%%
% Generic configuration.                                                       %
%%%%%%%%%%%%%%%%%%%%%%%%%%%%%%%%%%%%%%%%%%%%%%%%%%%%%%%%%%%%%%%%%%%%%%%%%%%%%%%%

% Color definitions.
\definecolor{bloodred}{RGB}{168, 43, 51}
\definecolor{royalblue}{RGB}{53, 171, 223}
\definecolor{mossgreen}{RGB}{34, 171, 66}
\definecolor{amber}{RGB}{233, 129, 45}
\definecolor{lightash}{RGB}{230, 230, 230}
\definecolor{darkash}{RGB}{211, 211, 211}

% Set up typography according to the typesetting engine being used.
\ifPDFTeX

  % Define the title font family to do nothing.
  \newcommand{\fflobster}{\relax}
\else

  % Use standard TeX ligatures.
  \defaultfontfeatures{Ligatures=TeX}

  % Define the title font family.
  \newfontfamily\fflobster{Lobster Two}

  % Set font style based on the Open Sans and Cousine fonts.
  \setmainfont[ItalicFont={Open Sans Light Italic},
               BoldFont={Open Sans SemiBold},
               BoldItalicFont={Open Sans SemiBold Italic}]
              {Open Sans Light}
  \setsansfont[ItalicFont={Open Sans Light Italic},
  						 BoldFont={Open Sans SemiBold},
  						 BoldItalicFont={Open Sans SemiBold Italic}]
  						{Open Sans Light}
  \setmonofont[ItalicFont={Cousine Italic},
  						 BoldFont={Cousine Bold},
  						 BoldItalicFont={Cousine Bold Italic}]{Cousine}
\fi

\makeatletter

% Customize title page.
\renewcommand{\maketitle}{

  \begin{center}

    % Draw title.
    {\LARGE\fflobster\color{bloodred}\@title}

    % Draw separator.
    \tikz[baseline={([yshift=-0.6em]current bounding box.center)}]{
      \draw[*-*, draw, color=bloodred] (0, 0) -- ++(\textwidth, 0);
    }

    % Add the author and date.
    \bigskip
    {\@author~$\cdot$~\@date}
  \end{center}

  \bigskip
}

\makeatother

% Set title style for sections. Note that with titlesec, anything that is placed
% around \thesection is placed around the section number in the rendered text.
\titleformat{\section}{\bfseries\large}{Task~\thesection~$\cdot$}{0.4em}{}

% Define new environment to delineate assignment questions.
\newenvironment{assignment}{
  \titleformat{\section}{\bfseries\large}{Task~\thesection*~$\cdot$}{0.4em}{}
}

\newenvironment{block}[1]{
  \begin{center}

    % Add vertical spacing between table rows.
    \renewcommand{\arraystretch}{1.5}

    % Create table to hold the contents of the block, highlighting the title
    % and body using two different colors.
    \begin{tabular}{p{0.9\textwidth}}
      \rowcolor{darkash}\textbf{#1}\\
      \rowcolor{lightash}}{\\
    \end{tabular}
  \end{center}
}

% Configure Hyperref colours.
\hypersetup{
  colorlinks,
  linkcolor=bloodred,
  citecolor=mossgreen,
  urlcolor=royalblue
}


%%%%%%%%%%%%%%%%%%%%%%%%%%%%%%%%%%%%%%%%%%%%%%%%%%%%%%%%%%%%%%%%%%%%%%%%%%%%%%%%
% Macro definitions.                                                           %
%%%%%%%%%%%%%%%%%%%%%%%%%%%%%%%%%%%%%%%%%%%%%%%%%%%%%%%%%%%%%%%%%%%%%%%%%%%%%%%%

% Latin and other abbreviations.
\newcommand*{\eg}{\textit{e.g.}\xspace}
\newcommand*{\Eg}{\textit{E.g.}\xspace}
\newcommand*{\ie}{\textit{i.e.,}\xspace}
\newcommand*{\etc}{\textit{etc.}\xspace}
\newcommand*{\corr}{\textit{corr.}\xspace}
\newcommand*{\etal}{\textit{et.~al}\xspace}
\newcommand*{\wrt}{w.r.t.\xspace}
\newcommand*{\cf}{\textit{cf.}\xspace}
\newcommand*{\nb}{\textit{n.b.}\xspace}
\newcommand*{\visavis}{\textit{vis-\`a-vis}\xspace}
\newcommand*{\resp}{resp.\xspace}
\newcommand*{\vs}{\textit{vs.}\xspace}
